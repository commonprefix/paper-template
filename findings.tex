\section{Findings}
\subsection{High}
    None Found.
\subsection{Medium}

\subsubsection*{M01: Extra hashing in \lstinline{ecvrf_encode_to_curve}}
\begin{itemize}[align=left]
    \item[\textbf{Affected Code:}] \href{https://github.com/MystenLabs/fastcrypto/blob/963205c6d0538fe548b8b10037cf87a53af6f424/fastcrypto/src/vrf.rs#L99}{fastcrypto/src/vrf.rs (line 99)}
    \item[\textbf{Summary:}] In the \lstinline{ecvrf_encode_to_curve} implementation, the \lstinline{expand_message} function applies the hash once before returning the expanded message. Additionally, the \lstinline{map_to_point} function applies the hash once again before calling the function \lstinline{from_uniform_bytes}. However, the \href{https://datatracker.ietf.org/doc/html/draft-irtf-cfrg-hash-to-curve-16#name-hashing-to-ristretto255}{Hash-to-Curve RFC Appendix B} only hashes once before calling the \lstinline{from_uniform_bytes} function. This does not lead to security issues but is a deviation from the specification and has negative performance implications.
    \item[\textbf{Suggestion:}] We recommend using \lstinline{from_uniform_bytes} directly in the \lstinline{ecvrf_encode_to_curve} to avoid the additional hashing and potential compatibility issues with other designs.
    \item[\textbf{Suggested Fix:}] \href{https://github.com/MystenLabs/fastcrypto/pull/543/commits/370694eeffed74a57c816758d472640e4011ae7e}{[370694]}
    \item[\textbf{Status:}] Open
\end{itemize}

\subsubsection*{M-02: Non-canonical ristretto scalar deserialization}
\begin{itemize}[align=left]
\item[\textbf{Affected Code:}] \href{https://github.com/MystenLabs/fastcrypto/blob/963205c6d0538fe548b8b10037cf87a53af6f424/fastcrypto/src/groups/ristretto255.rs#L191}{fastcrypto/src/groups/ristretto255.rs (line 191)}
\item[\textbf{Summary:}] The ristretto scalar deserialization differs from the one upstream such that it does not enforce the value to be canonical (i.e., that the input is already reduced). In the context of ECVRF, this is not dangerous as the s and c components of the proof should *not* be assumed unique. Proofs following the spec for nonce derivation are deterministic, but this cannot be checked or enforced by the verifier: an alternative prover can derive k via other means with no clear avenue for detection. In general use, however, it can enable multiple representations for elements that should be unique, making replay attack detection harder.
\item[\textbf{Suggestion:}] Consider updating the ristretto scalar deserialization to use upstream \lstinline{from_canonical_bytes function}.
\item[\textbf{Suggested Fix:}] \href{https://github.com/MystenLabs/fastcrypto/pull/543/commits/ebcd7e1c8e6f0bbf48667f995017151a20289886}{[ebcd7e1c]}
\item[\textbf{Status:}] Open
\end{itemize}

\subsection{Low}
\subsubsection*{L-01: Consider removing \lstinline{verify_output} and bring \lstinline{verify} inline with the RFC}
\begin{itemize}[align=left]
    \item[\textbf{Affected Code:}] \href{https://github.com/MystenLabs/fastcrypto/blob/963205c6d0538fe548b8b10037cf87a53af6f424/fastcrypto/src/vrf.rs#L45}{fastcrypto/src/vrf.rs (line 45)}
    \item[\textbf{Summary:}] The \lstinline{verify_output} function allows the verification of output given a proof, public key, input, and output. This function is not present in the RFC standard, and its usage could lead to potential pitfalls. The function requires the purported output of the VRF as an input. One can get the output in two ways: fetching it from someone or generating it locally by hashing the proof. The first scenario wastes bandwidth, as the output
    can be locally generated using the proof. In the second scenario, we generate the output twice: firstly, to provide it as an input to the function, and, secondly, inside the \lstinline{verify_output} itself. Removing the \lstinline{verify_output} function would harden the implementation, as its use case is unclear. Additionally, the current \lstinline{verify} function returns a boolean, while the RFC specification returns False or (True, Output).
    \item[\textbf{Suggestion:}] Consider removing the \lstinline{verify_output} function and modify the \lstinline{verify} function to align with the RFC specification, returning False or (True, Output). This will help prevent potential issues for downstream developers.
    \item[\textbf{Status:}] Open
\end{itemize}

\subsubsection*{L-02: Use of vartime multiscalar multiplication for faster verification}
\begin{itemize}[align=left]
\item[\textbf{Affected Code:}] \
\begin{itemize}
\item \href{https://github.com/MystenLabs/fastcrypto/blob/963205c6d0538fe548b8b10037cf87a53af6f424/fastcrypto/src/vrf.rs#L248}{fastcrypto/src/vrf.rs (line 248)}
\item \href{https://github.com/MystenLabs/fastcrypto/blob/963205c6d0538fe548b8b10037cf87a53af6f424/fastcrypto/src/vrf.rs#L252}{fastcrypto/src/vrf.rs (line 252)}
\end{itemize}
\item[\textbf{Summary:}] Multiscalar multiplication takes the most time during verification. The current implementation uses \lstinline{RistrettoPoint::multiscalar_mul} to ensure a constant runtime to avoid leaking information. However, the upstream \lstinline{RistrettoPoint} also allows for \lstinline{vartime_multiscalar_mul}, which is faster but does not run in constant time. \lstinline{vartime_multiscalar_mul} can be used for the verification side, where no private information can be leaked.
\item[\textbf{Suggestion:}] Consider using \lstinline{vartime_multiscalar_mul} for the verification side to improve performance without risking the leakage of private information.
\item[\textbf{Suggested Fix:}] \href{https://github.com/MystenLabs/fastcrypto/pull/543/commits/53e16c8f82d63a84cdc842d93fc5c26cebbe109b}{[53e16c]}
\item[\textbf{Status:}] Open
\end{itemize}

\subsection{Informational}
\subsubsection*{I-01: Choice of \lstinline{SUITE_STRING} and \lstinline{DST}}
\begin{itemize}[align=left]
\item[\textbf{Affected Code:}]\
\begin{itemize}
\item \href{https://github.com/MystenLabs/fastcrypto/blob/963205c6d0538fe548b8b10037cf87a53af6f424/fastcrypto/src/vrf.rs#L78}{fastcrypto/src/vrf.rs (line 78)}
\item \href{https://github.com/MystenLabs/fastcrypto/blob/963205c6d0538fe548b8b10037cf87a53af6f424/fastcrypto/src/vrf.rs#L88}{fastcrypto/src/vrf.rs (line 88)}
\end{itemize}
\item[\textbf{Summary:}] Currently, both \lstinline{SUITE_STRING} and \lstinline{DST} mention "sui". We understand that the fastcrypto library is primarily used for the Sui blockchain. We recommend using a more neutral identifier to encourage broader adoption of the implemented parameters.
\item[\textbf{Suggestion:}] Change the \lstinline{SUITE_STRING} and \lstinline{DST} identifiers to a more neutral term, such as "fastcrypto," to promote standardization and wider adoption of the library.
\item[\textbf{Status:}] Open
\end{itemize}

\subsubsection*{I-02: Consistent use of SHA512 reference type H}
\begin{itemize}[align=left]
\item[\textbf{Affected Code:}]\
\begin{itemize}
\item \href{https://github.com/MystenLabs/fastcrypto/blob/963205c6d0538fe548b8b10037cf87a53af6f424/fastcrypto/src/vrf.rs#L108}{fastcrypto/src/vrf.rs (line 108)}
\item \href{https://github.com/MystenLabs/fastcrypto/blob/963205c6d0538fe548b8b10037cf87a53af6f424/fastcrypto/src/vrf.rs#L137}{fastcrypto/src/vrf.rs (line 137)}
\item \href{https://github.com/MystenLabs/fastcrypto/blob/963205c6d0538fe548b8b10037cf87a53af6f424/fastcrypto/src/vrf.rs#L141}{fastcrypto/src/vrf.rs (line 141)}
\end{itemize}
\item[\textbf{Summary:}] Some parts of the code use the reference type H for SHA512, while others directly use SHA512. It is recommended to use the reference type H consistently throughout the code.
\item[\textbf{Suggestion:}] Modify the code to consistently use the reference type H of SHA512, ensuring uniformity and improved maintainability.
\item[\textbf{Suggested Fix:}] \href{https://github.com/MystenLabs/fastcrypto/pull/543/commits/d8edc92340dd957f4bd05c356379be7ffa0916a5}{[d8edc923]}
\item[\textbf{Status:}] Open
\end{itemize}

\subsubsection*{I-03: Dependency on curve25519-dalek-ng}
\begin{itemize}[align=left]
    \item[\textbf{Summary:}] The Ristretto255 implementation in the fastcrypto library depends on the \href{https://github.com/zkcrypto/curve25519-dalek-ng}{curve25519-dalek-ng} crate, which is a fork of the \href{https://github.com/dalek-cryptography/curve25519-dalek}{curve25519-dalek} crate. At the time of the audit, curve25519-dalek-ng is approximately 200 commits behind curve25519-dalek. This discrepancy may result in missed security updates, performance improvements, or other enhancements available in the original curve25519-dalek repository.
    \item[\textbf{Suggestion:}] We recommend reviewing the changes made in the curve25519-dalek repository since the fork and consider updating the dependency on curve25519-dalek-ng to include any relevant updates, bug fixes, or security patches. Alternatively, consider switching back to the original curve25519-dalek crate if the changes in curve25519-dalek-ng are no longer necessary or can be incorporated into the fastcrypto library in another way.
    \item[\textbf{Status:}] Open
\end{itemize}

\subsubsection*{I-04: Optimize challenge generation by caching hash state}
\begin{itemize}[align=left]
\item[\textbf{Affected Code:}] \href{https://github.com/MystenLabs/fastcrypto/blob/963205c6d0538fe548b8b10037cf87a53af6f424/fastcrypto/src/vrf.rs#L157}{fastcrypto/src/vrf.rs (line 157)}
\item[\textbf{Summary:}] The current implementation of \lstinline{ecvrf_challenge_generation} initializes and updates the hash with the constant \lstinline{SUITE_STRING} every time it is called. This can be optimized by caching the initial hash state to avoid redundant hashing of the constant \lstinline{SUITE_STRING}.
\item[\textbf{Suggestion:}] Introduce a cached hash state using \lstinline{Lazy} to store the initial state of the hash with the \lstinline{SUITE_STRING} already hashed. This would allow reusing the cached state and avoid the unnecessary hashing of \lstinline{SUITE_STRING} for each invocation of \lstinline{ecvrf_challenge_generation}.
\item[\textbf{Suggested Fix:}] \href{https://github.com/MystenLabs/fastcrypto/pull/543/commits/ae62e9bdb527574974a34413ef540d3f29d38d87}{[ae62e9]}
\item[\textbf{Status:}] Open
\end{itemize}

\subsubsection*{I-05: Store compressed state of ECVRFPublicKey to optimize challenge generation}
\begin{itemize}[align=left]
\item[\textbf{Affected Code:}] \href{https://github.com/MystenLabs/fastcrypto/blob/963205c6d0538fe548b8b10037cf87a53af6f424/fastcrypto/src/vrf.rs#L91}{fastcrypto/src/vrf.rs (line 91)}
\item[\textbf{Summary:}] The current implementation of \lstinline{ECVRFPublicKey} does not store its compressed state, leading to potential recalculations of the compressed form during challenge generation. This can be optimized by storing the compressed state of the \lstinline{ECVRFPublicKey} in memory.
\item[\textbf{Suggestion:}] Modify the \lstinline{ECVRFPublicKey} structure to include a field for the compressed representation. Update the serialization and deserialization methods to work with the new structure. This would allow for a more efficient challenge generation process by reusing the stored compressed state instead of recalculating it.
\item[\textbf{Suggested Fix:}] \href{https://github.com/MystenLabs/fastcrypto/pull/543/commits/a5bd0a335c4120fb05db801895a778905aaffcac}{[a5bd0a]}
\item[\textbf{Status:}] Open
\end{itemize}

\subsubsection*{I-06: Precompute multiscalar multiplication using VartimeRistrettoPrecomputation}
\begin{itemize}[align=left]
\item[\textbf{Affected Code:}] \href{https://github.com/MystenLabs/fastcrypto/blob/963205c6d0538fe548b8b10037cf87a53af6f424/fastcrypto/src/vrf.rs}{fastcrypto/src/vrf.rs}
\item[\textbf{Summary:}] Parts of multiscalar multiplication during the verification can be cached to improve the performance of the ECVRF verification process.
\item[\textbf{Suggestion:}] Introduce a \lstinline{challenge_cache} field in the \lstinline{ECVRFPublicKey} structure and use \lstinline{VartimeRistrettoPrecomputation} to precompute the multiscalar multiplication between the generator and the public key. 
\item[\textbf{Suggested Fix:}] \href{https://github.com/MystenLabs/fastcrypto/pull/543/commits/0c05200998a89fc177a26a8b9d4254bd0e0070de}{[0c0520]} (needs more work)
\item[\textbf{Status:}] Open
\end{itemize}
